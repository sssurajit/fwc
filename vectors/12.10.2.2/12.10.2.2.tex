\documentclass[12pt]{article}
\usepackage{graphicx}
\graphicspath{{./figs/}}{}
\usepackage{amsmath,amssymb,amsfonts,amsthm}
\newcommand{\myvec}[1]{\ensuremath{\begin{pmatrix}#1\end{pmatrix}}}
\providecommand{\norm}[1]{\lVert#1\rVert}
\usepackage{listings}
\usepackage{watermark}
\usepackage{titlesec}
\usepackage{caption}
\let\vec\mathbf
\lstset{
frame=single, 
breaklines=true,
columns=fullflexible
}
\thiswatermark{\centering \put(0,-105.0){\includegraphics[scale=0.15]{logo.png}} }
\title{\mytitle}
\title{
Assignment - 12.10.2.2
}
\author{Surajit Sarkar}
\begin{document}
\maketitle
\tableofcontents
\bigskip
\section{\textbf{Problem}}
Write two different vectors having same\\ magnitude
\section{\textbf{Solution}}
\begin{align}
\text{Cosider}\\
\overrightarrow{a} &=2\hat{i}+3\hat{j}+k \\ 
\overrightarrow{b}&=3\hat{i}+2\hat{j}-k\\ 
\end{align}
\begin{enumerate}
    \item 
\begin{align}
\|\vec{A}\|=|\overrightarrow{\vec{A}}|&=\sqrt{\vec{A}^{\top}\vec{A}} \\
&=\sqrt{\myvec{2&3&1}\myvec{2\\3\\1}}\\
&=\sqrt{4+9+1}\\
&=\sqrt{14}
\end{align}
\item
\begin{align}
\|\vec{B}\|=|\overrightarrow{\vec{B}}|&=\sqrt{\vec{B}^{\top}\vec{B}} \\
&=\sqrt{\myvec{3&2&(-1)}\myvec{3\\2\\(-1)}}\\
&=\sqrt{4+9+1}\\
&=\sqrt{14}
\end{align}
\end{enumerate}
\end{document}

