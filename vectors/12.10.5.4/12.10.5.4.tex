\documentclass[journal,12pt,twocolumn]{IEEEtran}
\usepackage{graphicx}
\graphicspath{{./figs/}}{}
\usepackage{amsmath,amssymb,amsfonts,amsthm}
\newcommand{\myvec}[1]{\ensuremath{\begin{pmatrix}#1\end{pmatrix}}}
\providecommand{\norm}[1]{\lVert#1\rVert}
\usepackage{listings}
\usepackage{watermark}
\usepackage{titlesec}
\usepackage{caption}
\newcommand{\mydet}[1]{\ensuremath{\begin{vmatrix}#1\end{vmatrix}}}
\let\vec\mathbf
\lstset{
frame=single, 
breaklines=true,
columns=fullflexible
}
\thiswatermark{\centering \put(0,-105.0){\includegraphics[scale=0.15]{/sdcard/IITH/vectors/12.10.5.4/figs/logo.png}} }
\title{\mytitle}
\title{
Assignment - 12.10.5.4
}
\author{Surajit Sarkar}
\begin{document}
\maketitle
\tableofcontents
\bigskip
\section{\textbf{Problem}}
if $\vec{a}=\vec{b}+\vec{c}$, then is true that $|\vec{a}|=|\vec{b}|+|\vec{c}|$?
Justify your answer .
\section{\textbf{Solution}}
Given
\begin{align}
    \vec{a}=\vec{b}+\vec{c}
\end{align}
let
\begin{align}
    \vec{b}=\myvec{1\\2\\3},\vec{c}=\myvec{2\\-1\\-2}
\end{align}
thus
\begin{align}
    \vec{a}&=\vec{b}+\vec{c}\\
    &=\myvec{3\\1\\1}
\end{align}
\begin{align}
    \norm{\vec{a}}&=\sqrt{\vec{a}^{\top}\vec{a}}\\
    &=\sqrt{\myvec{3&1&1}\myvec{3\\1\\1}}\\
    &=\sqrt{9+1+1}\\
    &=\sqrt{11}
\end{align}
\begin{align}
    \norm{\vec{b}}&=\sqrt{\vec{b}^{\top}\vec{b}}\\
    &=\sqrt{\myvec{1&2&3}\myvec{1\\2\\3}}\\
    &=\sqrt{1+4+9}\\
    &=\sqrt{14}
\end{align}
\begin{align}
    \norm{\vec{c}}&=\sqrt{\vec{c}^{\top}\vec{c}}\\
    &=\sqrt{\myvec{2&-1&-2}\myvec{2\\-1\\-2}}\\
    &=\sqrt{4+1+4}\\
    &=\sqrt{9}\\
    &=3
\end{align}
Now
\begin{center}
    $|\vec{a}|\neq|\vec{b}|+|\vec{c}|$
\end{center}
\begin{lstlisting}
https://github.com/sssurajit/fwc/blob/main/vectors/12.10.5.4/codes/code.py
\end{lstlisting}
Execute the code by using the command\\
\textbf{python3 code.py}
\end{document}

