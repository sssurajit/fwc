\documentclass[journal,12pt,twocolumn]{IEEEtran}
\usepackage{amsmath,amssymb,amsfonts,amsthm}
\newcommand{\myvec}[1]{\ensuremath{\begin{pmatrix}#1\end{pmatrix}}}
\providecommand{\norm}[1]{\lVert#1\rVert}
\usepackage{listings}
\usepackage{watermark}
\usepackage{titlesec}
\usepackage{caption}
\usepackage{gensymb}
\newcommand{\mydet}[1]{\ensuremath{\begin{vmatrix}#1\end{vmatrix}}}
\let\vec\mathbf
\lstset{
frame=single, 
breaklines=true,
columns=fullflexible
}
\title{\mytitle}
\title{
Assignment - 11.10.2.8
}
\author{Surajit Sarkar}
\begin{document}
\maketitle
\tableofcontents
\bigskip
\section{\textbf{Problem}}
The perpendicular distance from the origin is 5 units and the angle made by the perpendicular with the positive x-axis is $30^{\circ}$.Find the equation of the line ?
\section{\textbf{Solution}}
\begin{align}
\vec{m}&=\tan30^{\circ}\\
\vec{m}&=\myvec{1\\\frac{1}{\sqrt{3}}}
\end{align}
\begin{align}
d&=\frac{|\vec{c}|}{\norm{\vec{m}}}\\
\vec{c}&=\frac{10}{\sqrt{3}}
\end{align}
Equation
\begin{align}
    \vec{m}^{\top}\vec{X}&=\vec{c}\\
    \myvec{1&\frac{1}{\sqrt{3}}}\vec{X}&=\frac{10}{\sqrt{3}}\\
    \myvec{\sqrt{3}&1}\vec{X}&=10\\
    \sqrt{3}\vec{x}+\vec{y}&=10
\end{align}
\end{document}

