% #######################################
% ########### FILL THESE IN #############
% #######################################
\def\mytitle{Karnaugh-map Using Arduino}
\def\mykeywords{}
\def\myauthor{SURAJIT SARKAR}
\def\contact{sarkars531@gmail.com}
\def\mymodule{FUTURE WIRELESS COMMUNICATIONS-(FWC22085)}
% #######################################
% #### YOU DON'T NEED TO TOUCH BELOW ####
% #######################################
\documentclass[10pt, a4paper]{article}
\usepackage[a4paper,outer=1.5cm,inner=1.5cm,top=1.75cm,bottom=1.5cm]{geometry}
\twocolumn
\usepackage{graphicx}
\graphicspath{{./images/}}
%colour our links, remove weird boxes
\usepackage[colorlinks,linkcolor={black},citecolor={blue!80!black},urlcolor={blue!80!black}]{hyperref}
%Stop indentation on new paragraphs
\usepackage[parfill]{parskip}
%% Arial-like font
\usepackage{lmodern}
\renewcommand*\familydefault{\sfdefault}
%Napier logo top right
\usepackage{watermark}
%Lorem Ipusm dolor please don't leave any in you final report ;)
\usepackage{karnaugh-map}
\usepackage{tabularx}
\usepackage{lipsum}
\usepackage{xcolor}
\usepackage{listings}
%give us the Capital H that we all know and love
\usepackage{float}
%tone down the line spacing after section titles
\usepackage{titlesec}
%Cool maths printing
\usepackage{amsmath}
%PseudoCode
\usepackage{algorithm2e}

\titlespacing{\subsection}{0pt}{\parskip}{-3pt}
\titlespacing{\subsubsection}{0pt}{\parskip}{-\parskip}
\titlespacing{\paragraph}{0pt}{\parskip}{\parskip}
\newcommand{\figuremacro}[5]{
    \begin{figure}[#1]
        \centering
        \includegraphics[width=#5\columnwidth]{#2}
        \caption[#3]{\textbf{#3}#4}
        \label{fig:#2}
    \end{figure}
}

\lstset{
frame=single, 
breaklines=true,
columns=fullflexible
}
\thiswatermark{\centering \put(30,-80.0){\includegraphics[scale=0.2]{logo2}} }
\title{\mytitle}
\author{\myauthor\hspace{1em}\\\contact\\IITH\hspace{0.5em}-\hspace{0.5em}\mymodule}
\date{}
\hypersetup{pdfauthor=\myauthor,pdftitle=\mytitle,pdfkeywords=\mykeywords}
\sloppy
% #######################################
% ########### START FROM HERE ###########
% #######################################
\begin{document}
 \maketitle
 \tableofcontents
 \begin{abstract}
The objective of this manual is to show how to \\verify 
following min-terms.
    
    f(A,B,C,D) = $\sum m(2,3,8,10,11,12,14,15)$
    
using karnaugh-map
    
 \end{abstract}
    
 

 \section{Introduction}
    Karnaugh-map provides a systematic method for
    \\simplifying boolean expressions and may produce
    \\simplest SOP or POS expressions.
    
    
    karnaugh-map used to minimize number of logic
    \\gates that are required in a digital circuit.
    
    
    
    
    
    \section{components}
    \begin{tabularx}{0.4\textwidth} { 
  | >{\centering\arraybackslash}X 
  | >{\centering\arraybackslash}X 
  | >{\centering\arraybackslash}X 
  | >{\centering\arraybackslash}X | }
  \hline
  component & value & quantity \\
  \hline
  Arduino & UNO & 1 \\
  \hline
  
  \end{tabularx}
  \begin{center}
      Table-0
  \end{center}
  
  
  

    
    
 
 
 \section{karnaugh-map}
 \subsection{Implementation}
 \begin{karnaugh-map}[4][4][1][$CD$][$AB$]
    \minterms{2,3,8,10,11,12,14,15}
    \maxterms{0,1,2,3,4,5,6,7,8,9,10,11,12,13,14,15}
     \implicant{15}{10}
        \implicantedge{3}{2}{11}{10}
        \implicantedge{12}{8}{14}{10}
    \end{karnaugh-map} 
    \begin{center}
    Figure 1:k-map
    \end{center}
       
        From the above karnaugh-map the expression is
    
    
      
    f=$ A\overline{D}+AC+\overline{B}C$
 
  
          
       
       This karnaugh-map is verified by using 
       
       
       Truthtable Table-1
       
\section{Truthtable}
  \begin{tabularx}{0.5\textwidth} { 
  | >{\centering\arraybackslash}X 
  | >{\centering\arraybackslash}X 
  | >{\centering\arraybackslash}X 
  | >{\centering\arraybackslash}X
  | >{\centering\arraybackslash}X |}
  \hline
  A & B & C & D & O/P \\
 
   \hline
  0 & 0 & 0 & 0 & 0  \\
   
   \hline
  0 & 0 & 0 & 1 & 0  \\
 
   \hline
  1 & 0 & 1 & 0 & 1  \\

   \hline
  1 & 0 & 1 & 1 & 1  \\
   \hline
  0 & 1 & 0 & 0 & 0  \\
   \hline
  0 & 1 & 1 & 0 & 0  \\
 
   \hline
  0 & 1 & 1 & 1 & 0  \\
   \hline
  1 & 0 & 0 & 0 & 1  \\
  \hline
  
  \end{tabularx}
  \begin{center}
      Table-1
  \end{center}
\section{Hardware Connections}
1.connect the arduino to the computer




  
2.The led will ON and OFF when changing the inputs .



.



\section{Software}
  Download the follwing code
  
 


 \begin{lstlisting}
  https://github.com/sssurajit/fwc/blob/main/codes/src/main.cpp
  \end{lstlisting}


       
  
\end{document}
  
\end{document}


